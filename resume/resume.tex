\documentclass[11pt]{res} % default is 10 pt

\usepackage{helvetica} % uses helvetica postscript font (download helvetica.sty)
\usepackage[colorlinks,urlcolor=blue]{hyperref}

\setlength{\textheight}{9.5in} % increase text height to fit resume on 1 page
\newsectionwidth{0pt}  % So the text is not indented under section headings

\begin{document} 
 
\name{Frank Pan/Zhenhao Pan\\[12pt]} % the \\[12pt] adds a blank line
                                     % after name

\address{{\bf PRESENT ADDRESS} \\ Room 503, 26\# \\ Tsinghua
  University \\ Beijing China}

\address{{\bf CONTACTS} \\ Email:
  \href{mailto:frankpzh@gmail.com}{frankpzh@gmail.com} \\ Mobile:
  (+86) 1342-630-6840 \\ Blog: \href{http://frankpzh.wordpress.com/}{http://frankpzh.wordpress.com/}}

\begin{resume}

\section{SKILLS}

\begin{itemize} \itemsep -2pt
\item Systems: Linux kernel, Xen, Windows Research Kernel, Windows CE, Android, Symbian
\item Languages: C, Python, Makefile, C++, Java, Elisp, Bash, Javascript, VHDL
\item Tools: git, svn, emacs, vim, grep/sed/awk, binutils
\end{itemize} 

\section{INTERNSHIPS}
Software Engineer, \textbf{Intel} China, Shanghai China \hfill Mar.2010 to Feb.2011
\begin{itemize} \itemsep -2pt
\item I finished live migration support for SR-IOV network device on Xen on my own.
\item The idea is saving/restoring the internal state of a network device. (c, linux kernel)
\item To complete the job, I hacked Xen to support migration with pass-through device. (c, python, xen)
\item In addition, rapid checkpointing (high availability, Remus) is also supported.
\item I also work with Intel full-time employees on 2 academic papers, listed in "publications" section.
\end{itemize} 

Software Engineer, Next Mobile Web, Mountain View, Remotely from China \\
\hspace*{\fill} Jun.2009 to Sep.2009
\begin{itemize} \itemsep -2pt
\item During the internship, I completed tab bar support of Android Web Browser (Java, Android 1.5)
\end{itemize} 

Software Engineer, \textbf{Microsoft Research Asia}, Beijing China \hfill May.2008 to Aug.2008
\begin{itemize} \itemsep -2pt
\item Kernel code injector (KInjectToolkit) for Windows Research Kernel (WRK) is my achievement.
\item It dynamically loads pre-compiled binary code from user space into kernel space. (c, WRK)
\item A toolkit is designed for generating binary code from c source programs. (c, PE format)
\item It has collected by Aimin Pan's book as a tool
  (\href{http://www.phei.com.cn/windowskernelbook/zhu\_ye.html}{http://www.phei.com.cn/windowskernelbook/zhu\_ye.html})
\end{itemize} 

\section{PROJECT EXPERIENCES}
Chinese character support on frame buffer console \hfill May.2010

\begin{itemize} \itemsep -2pt
\item Hack Linux kernel on console driver (c, UTF-8)
\item Load font data from user space
\end{itemize}

QQReader: Ebook reader on Symbian S60  (c++, symbian) \hfill Sep.2009 to Dec.2009

\begin{itemize} \itemsep -2pt
\item E-book format: TXT, TCR, UMD, PDB, RSS, ZIP
\item Several UI optimizations
\item It's the prototype of the incoming mobile e-book reader of Tencent™
\end{itemize}

Porting Windows CE on OpenMoko cell phone \hfill Feb.2009 to Mar.2009

\begin{itemize} \itemsep -2pt
\item Completed a board support package (BSP) and drivers necessary (c, ARM)
\item OpenMoko (Neo FreeRunner): Samsung 2442B SOC (ARM920T core), S-Media 3362
\item Bachelor's thesis
\end{itemize}

StockHamster: A simulator of a stock market (Teamwork) \hfill Oct.2008 to Dec.2008

\begin{itemize} \itemsep -2pt
\item The system runs with 50 AIs and a rumor agency (Java)
\item With a console client and a swing client to monitor the market
\item Worked as a team leader, completed the design of system
\item Course project
\end{itemize}

Multimedia Computer on FPGA (Teamwork) \hfill Oct.2007 to Jan.2008

\begin{itemize} \itemsep -2pt
\item Designed a four-step pipelining structure
\item Implemented a processor in VHDL
\item Drove several peripherals including audio and video output
\item Processor frequency can be up to 50MHz on Cyclone II FPGA
\item Worked as a team leader
\item Course project
\end{itemize}

%Music Player on FPGA (VHDL) \hfill Mar.2007 to Jun.2007

%\begin{itemize} \itemsep -2pt
%\item Purely hardware implementation, no processor
%\item Support FAT file system, WAV audio format
%\item Peripherals: SD-Card, VGA display, keyboard, and WM8731 audio output
%\item No external library used
%\item Course project
%\end{itemize}

\section{PUBLICATIONS}
NestCloud: Towards Pratical Nested Virtualization, \emph{2011 IEEE
  International Conference on Cloud and Service Computing (CSC2011)}

ReNIC: Architecture Extension to SR-IOV for Efficient Replication,
\emph{7th International Conference on High-Performance and Embedded
  Architectures and Compilers (HiPEAC 2012)}, pending.

CompSC: Live Migration with pass-through devices, \emph{The 2012 ACM
  SIGPLAN/SIGOPS International Conference on Virtual Execution
  Environments (VEE 2012)}, pending.

\section{REVIEWS}
Review of \textbf{Cloud Computing in Action}, by Rod Cope, not published yet.

\section{EDUCATIONS}
Master's degree in Computer Science and Technology, Tsinghua University \\
\hspace*{\fill} Aug.2009 to Jan.2012

Bachelor's degree in Computer Science and Technology, Tsinghua University \\
\hspace*{\fill} Aug.2005 to Jul.2009
\begin{quote}
Overall GPA: 86.14/100, Major GPA: \textbf{92.1/100} \\
Tsinghua University Freshman Scholarship \\
Tsinghua University Annual Scholarship in 2007, 2008 \\
\end{quote}

\section{AWARDS}
2008 "Computer World" Award \hfill Dec.2008 \\
2007 Nios II Embedded Processor Design Contest, Outstanding Award \hfill Aug.2007 \\
2004 National Olympiad in Informatics, 3rd place \hfill Aug.2004

\end{resume}
\end{document}
